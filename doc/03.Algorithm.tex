\chapter{Алгоритм работы}

\begin{enumerate}
\item Получение исходного текста из оболочки
\item Составление карты ударений
\item Поиск рифмованных строк
\item Замена фразеологизмов
\item Пословный перевод — слова складываются в стек
\item Оформление перевода
\begin{enumerate}
 \item Деление карты ударений на строки. Начало цикла по строкам.
 \item Установка счётчика на начало строки.
 \item Начало цикла по счётчику. Цикл продолжается, пока счётчик не пройдёт конец строки.
 \item Получение слова из стека переведённых слов.
 \item Получение всех словоформ этого слова.
 \item Если слово — последнее или предпоследнее в строке, то отсеивание всех не рифмующихся (или плохо рифмующихся) форм. Если все формы не подходят — отсеивание не используется.
 \item Вычисление для каждой формы различия с отрезком карты ударений.
 \item Любой (первый) вариант из подходящих записывается в перевод.
 \item Счётчик передвигается вперёд на количество слогов в поставленной форме. Конец цикла по счётчику.
 \item Вывод новой строки в перевод. Конец цикла по строкам.
\end{enumerate}
\item Вывод результата в оболочку
\end{enumerate}

Возможна паралеллизация пунктов 2 и 3, а также на пунктах 1, 2, 4 и 5.

\subsection{Составление карты ударений}

Так как ударения должны быть уже проставлены в исходном тексте, этот этап алгоритма тривиален. Текст сканируется на гласные; при обнаружении гласной с проставленным ударением, или знака ударения\footnote{Апостроф не считается знаком ударения. Во-первых, потому что он может использоваться в тексте, во-вторых, потому что программа работает в кодировке UTF-8 и может использовать все знаки этой кодировки.} в карту ставится знак «!», при обнаружении безударной гласной — знак «-». Пробелы не отслеживаются, переносы строк записываются.

Важно отметить,что ритм не обязан быть чётким, например:

\begin{verse}
Ярко солнце светит,\\
Щебечет воробей.\\
Добрым жить на белом свете\\
Веселей.
\end{verse}
\hskip 6cm \textit{(А. Хайт)}

Именно поэтому анализ идёт построчно.

\subsection{Поиск рифмованных строк}

Используются два вложенных цикла по строкам. В каждом из них находятся ударные гласные последних слов строк. Если гласные совпадают, то сравниваются также ударные гласные предпоследних слов строк. Количество совпадающих гласных записывается в хэш по номерам строк.

\subsection{Замена фразеологизмов}
Замена идиом, фразеологизмов и сленга на общеупотребительную лексику. К этому моменту ритм уже должен быть вычислен, а рифмы — найдены.

\subsection{Определение рифмы}

Один из самых сложных алгоритмов --- алгоритм подбора рифмы. Здесь нужно остановиться и описать варианты поподробнее. Не будем рассматривать внутреннюю рифму, когда рифмуются также середины строк, так как в этом случае каждую строку можно разбить на две.

\subsubsection{Точное совпадение}

В самом абстрактном приближении рифма --- это совпадение последних букв в строках:

\begin{verse}
Упал со стола ботинок,\\
За ним --- полуботинок.\\
Украл ботинки инок.
\end{verse}

Как видно из примера, этот алгоритм имеет существенный недостаток: он может зарифмовать все строки одним словом или его производными. Это плохо читается, это очень сильно порицается поэтами и это практически относится к тавтологии. Такие стихи могут существовать, но они все будут очень низкой пробы. Кроме того, в этом алгоритме нет никакой фантазии. Как известно, в русском языке есть несколько слов, не имеющих рифмы вообще (т.н. арифматы), например, <<ж\'{и}вопись>>. Подобный алгоритм будет просто повторять эти слова из строки в строку.

\subsubsection{Точное совпадение гласных}

Этот подход чуть вольнее предыдущего; в нём имеют значение только \textbf{гласные} в окончаниях строк. Гласные должны совпадать, совпадение же согласных необязательно. Вот демонстрация такого алгоритма:

\begin{verse}
Трам-парам-парам-опилки!\\
Пум-шурум-бурум-шуминги!
\end{verse}

Это был вариант среднего удобочитания. Есть и похуже:

\begin{verse}
Всё будет сегодня отлично!\\
Нашёл я коробку ботинков!
\end{verse}

Даже если потребовать,чтобы рифмующиеся слова имели одинаковую длину и окончания, рифма <<отлично---ботинко>> не совсем радует слух. Поэтому этот алгоритм имеет ещё более серьёзный недостаток: он очень редко работает. Удачная рифма в таком алгоритме --- не более,чем совпадение; действительно, согласных в словах намного больше,чем гласных, и получить рифмующиеся слова будет очень сложно.

Для ускорения работы можно потребовать совпадения лишь ударных гласных. Это и используется в программе.

\subsubsection{Созвучие букв в слогах}

Рифма всегда определяется произношением. В любой рифме окончания строк всегда созвучны. Но при этом рифмы принято делить на точные и неточные.

При использовании точной рифмы созвучными оказываются не только ударные звуки в окончаниях строк, но и слоги, расположенные за ними. Неточная рифма характеризуется различиями в звучании согласных в свободных от ударения слогах, расположенных в окончаниях строк.

Пример точной рифмы:

\begin{verse}
Я встретил слонопопотама\\
С фамилией Ятутатама...
\end{verse}

Пример неточной рифмы:

\begin{verse}
Я встретил слонопопотама\\
С фамилией Нипокадрано...
\end{verse}

В этих примерах очень мало смысла, но они прекрасно демонстрируют принцип рифмовки.

Ударные слоги должны быть созвучными. Совпадение --- частный случай созвучия.

Чем дальше ударный слог рифмы от конца строки, тем менее важно всё,что стоит перед ним. В трёхсложных и четырёхсложных рифмах созвучие пятых слогов от конца уже не так важно. При этом созвучие ударных слогов должно быть точным, а для безударных это необязательно.  Если слоги, стоящие после ударного, малосозвучны, это тоже считается рифмой (хотя и плохой). Но важно созвучие всего ударного слога, а не только гласной в нём.

Созвучие проще всего определить при помощи видовых и родовых пар. Видовые пары --- это взаимозаменяемые гласные звуки в рифме. Каждая видовая пара состоит из твёрдой и мягкой гласной.
\begin{itemize}
\item А-Я
\item О-Ё
\item У-Ю
\item Э-Е
\item Ы-И
\end{itemize}

Любой ударный твёрдый гласный  звук рифмуется с его мягкой парой, и наоборот.

Согласные звуки объединены в родовые пары и тройки:

\begin{itemize}
\item Б-П
\item В-Ф
\item Д-Т
\item Ж-Ш
\item З-С
\item К-Г-Х
\item Ц-Ч-Щ
\end{itemize}

Звонкая согласная в контексте рифмы всегда рифмуется с её глухой парой, и наоборот.

Насчёт троек следует оговориться. К-Х, как и Ц-Щ, не являются парными согласными. Рифмы с ними допустимы при определённых обстоятельствах, но они не будут высоки по качеству:

\begin{verse}
Как тогда один пустяк\\
Уничтожил в один взмах.
\end{verse}

\begin{verse}
Как из города Елец\\
К нам пришёл огромный клещ
\end{verse}

Поэтому при реализации лучше опираться на \textbf{пары} согласных, разбивая тройки надвое.

Этот алгоритм также не гарантирует защиту от повторения арифматов, но он имеет б\'{о}льшую гибкость в подборе, а значит, и меньшую склонность к тавтологии.

\subsubsection{Созвучие слогов}

Предыдущий алгоритм хорош, но всё-таки не выдерживает критики. Почему для определения созвучия слогов используются буквы, а не звуки? Логичнее использовать другой подход: составлять звуковые транскрипции слов и определять созвучие по ним. Этот метод точнее, но также сложнее. Количество звуков в слове чаще всего не равно количеству букв, и гласные буквы могут выражаться несколькими звуками. Следовательно, разбиение слова по слогам несколько затрудняется.

Основных звуков 43: [а э и о у ы п п' б б' м м' ф ф' в в' т т' д д' н н' с с' з з' р р' л л' ш ж щ җ ц ч й к к' г г' х х']

Также стоит выделять звуки с оттенками: так, в слове <<йети>> транскрипция выглядит как [йэт’и$^э$].  Оттенков может быть шесть, и они применяются ко всем согласным. То есть, общее число звуков равно 6+1+35*7=252.

Но это -- всего лишь мелкое неудобство. Дело в том,что сейчас нет верного алгоритма для автоматической расстановки транскрипции. Более того: многие слова в словаре имеют несколько верных транскрипций. Таким образом, необходимо составлять словарь произношений, в связи с чем сложность реализации алгоритма по сравнению со сложностью реализации предыдущего алгоритма значительно возрастает\footnote{Можно отказаться от оттенков - в таком случае автотранскрипция возможна, но это снизит возможности для перебора рифм.}.

Защиты от повторения арифматов алгоритм не имеет.


Сложность этого этапа алгоритма состоит в огромном количестве комбинаций и перестановок, которые потребуется просчитать.

\subsection{Разбивка русского слова на слоги}

В одном слоге не может быть больше одной гласной.

\subsection{Расстановка ударения на русском}

В русском языке слово может иметь, помимо основного, второстепенное (побочное) ударение ~\cite{udar}.

Если в слове есть буква <<ё>>, на неё всегда ставится ударение.

В односложных словах ударение ставится однозначно, если это не служебное слово или частица. Односложные предлоги и союзы чаще являются безударными словами.

Также возможен перенос ударения на предлог (н\'{а} воду), если существительное не имеет достаточной важности в тексте. Перенос возможен для односложных предлогов; чаще всего это НА, ЗА, ПОД, ПО, ИЗ, БЕЗ. Ударение не переносится на предлог в сочетании с числительными ~\cite{orfo}.

Из двусложных предлогов безударными являются ИЗ-ЗА, ИЗ-ПОД, ПОДО, НАДО, ОБО, ОТО, ИЗО.

Слабоударяемые слова\footnote{После, кругом, мимо, вокруг, напротив, поперёк, около и др. ~\cite{rosental}} являются особенностью русского языка и в исследование не берутся, так как ритм стихотворения берётся из перевода.

Вычисленные частоты можно посмотреть в коде программы. Из списков убраны высокочастотные шумы.

Разбивка на слоги производилась автоматически в соответствии с указанными правилами. Всего было обработано 20140 слов.

Относительные частоты высчитывались от всего количества слогов из данной группы (под ударением, без ударения, побочное ударение).

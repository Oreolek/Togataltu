\chapter{Краткое теоретическое вступление}
\subsection{Основные понятия}
Существует два основных типа стихотворных переводов:

\begin{enumerate}
\item Перестраивающий (содержание, форму)
\item Воссоздающий --- то есть, воспроизводящий с возможной полнотой и точностью содержание и форму.
\end{enumerate}

Первый вид перевода также часто называют вольным. Он не так высоко ценится, как воссоздающий (точный) перевод, но легче выполняется человеком (так как позволяет не следить за точным совпадением содержания и формы).

Для машинного перевода более подойдёт точный тип поведения, так как он наиболее формализован.

Важные отличия стихотворной формы от прозаической -- это наличие ритма и чёткого разбиения на строки, строфы и стихи. И хотя наличие ритма не является обязательным (так, у некоторых народов мира он может отсутствовать; кроме того, некоторые авторы намеренно отходят от ритма), будем рассматривать его как необходимый признак стихотворения.

Ритмичность складывается из чередования ударных и безударных слогов, тем более, что ударение в русском языке подвижно (к противному примеру, французский язык имеет ударение только на последний слог).

Ритм обычно един для всего стихотворения и позволяет лишь незначительные отклонения. Тем не менее, слова также имеют ударные и безударные слоги -- например, в литературном русском нет слова <<\'{о}кно>> с ударением на первый слог. На безударный слог нельзя ставить ударение, иначе это может сильно повредить прочтению вслух. Важно оговориться, что ударные слоги напротив, могут стать безударными.\label{distance}

Итак, перевод должен максимально точно сохранять форму и содержание исходного документа.

\bigskip

%Вариантов этих определений так много, что я смело взял самые удобные.

\begin{description}
\item[Фраза] --- это законченное высказывание.
\end{description}

Как \textbf{предложение} удобно понимать последовательность идущих друг за другом фраз, возможно, соединённых сочинительным союзом.

Фразы являются смысловыми единицами -- например, они могут быть однородными; в этом предложении их три.

\bigskip

\begin{description}
\item[Стопа] --- постоянная единица деления стиха, определяющая его метр. В силлабо-тонике это повторяющееся в пределах стихотворной строки сочетание одного ударного слога и определённого числа безударных слогов, \cite{literature-rosman}
\item[Вольный стих] --- форма стихотворного произведения, в которой все строки выдержаны в едином силлаботоническом метре, но неравностопны, \cite{literature-rosman}
\item[Метр] --- абстрактная, предельно общая ритмическая схема для каждой из строк стихотворного произведения, \cite{literature-rosman}
\item[Размер] --- абстрактная ритмическая схема, которой подчиняется индивидуальный ритм каждой строки отдельного стихотворного текста. Размер определяют в зависимости от системы стихосложения. В силлабике – по постоянному числу слогов, составляющих каждый стих. В силлабо-тонике – по постоянному числу стоп конкретного метра во всех строках текста или в тех, что занимают определённые позиции в каждой его строфе. В тонике размер определяют по постоянному числу ударений, \cite{literature-rosman}
\item[Рифма] --- полное или частичное совпадение звуков в конце стиха, начиная с последней ударной гласной.
\end{description}

Я не буду приводить теорию по переводу, так как в реализованной программе используется простая пословная подстановка.